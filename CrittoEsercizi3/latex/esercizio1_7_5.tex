Implementa l'algoritmo Pohlig-Hellman per trovare il logaritmo discreto in ${\mathbb{Z}_p}^*$. Con $p$ primo e $\alpha$ un elemento primitivo modulo $p$. Trova $\log_{5}(8563)$ in ${\mathbb{Z}_{28703}}$ e $\log_{10}(12611)$ in ${\mathbb{Z}_{31153}}$

\subsection*{Soluzione}
Per procedere con l'implementazione dell'algoritmo sono necessarie alcune funzioni correttamente implementate:
\begin{itemize}
    \item Algoritmo di fattorizzazione
    \item Algoritmo di \textit{Shanks} per il calcolo di logaritmi discreti
\end{itemize}

L'algoritmo di fattorizzazione è visibile nel file \texttt{Crittoesercizi3/Esercizio 1/PohligHellman.py}, per semplicità e stato scritto in modo ricorsivo, e fa uso della programmazione dinamica per il calcolo immediato della fattorizzazione per input precedentemente visti.\\
L'output è una lista di tuple $(p_0,c_0),\dots,(p_k,c_k)$ tale che  
\[
    n = \prod_{i = 0}^{k}{ {p_i}^{c_i}}
\]

Una volta ottenuta la fattorizzazione di $n$, il resto dell'implementazione, segue passo passo la teoria. Difatti il codice principale fa uso della fattorizzazione di $n$ per il calcolo di logaritmi discreti $a = \log_{\alpha}(\beta)$\\
con $a\mod {p_i}^{c_i} \; \forall i = 1,\dots,k$.

\subsubsection*{Calcolo logartimo discreto}
Si descrivono brevemente i dati del primo punto dell'esercizio $\log_{5}(8563)$ in ${\mathbb{Z}_{28703}}$, per cui $\alpha = 5,n = 28702, \beta = 8563$. La fattorizzazione di $n$ è $28702 = 113^1 \cdot 127^1 \cdot 2^1$.

\noindent
Test per il fattore $q = 113$ $c = 1$
\begin{center}
    \begin{tabular}{@{}lccl@{}}
        \toprule
        Step & Variabile & & Valore \\ 
        \midrule
                0 & $\beta_{0}$ & $=$ &$8563$\\
        &$\delta$ & $=$ & ${8563^{28702/{113}}}$\\
        &$a_{0}$ & $=$ & $67$\\
        &$\beta_{1}$ & $=$ & $17744$\\
        \bottomrule
    \end{tabular}\end{center}
Test per il fattore $q = 2$ $c = 1$
\begin{center}
    \begin{tabular}{@{}lccl@{}}
        \toprule
        Step & Variabile & & Valore \\ 
        \midrule
                0 & $\beta_{0}$ & $=$ &$8563$\\
        &$\delta$ & $=$ & ${8563^{28702/{2}}}$\\
        &$a_{0}$ & $=$ & $1$\\
        &$\beta_{1}$ & $=$ & $24675$\\
        \bottomrule
    \end{tabular}\end{center}
Test per il fattore $q = 127$ $c = 1$
\begin{center}
    \begin{tabular}{@{}lccl@{}}
        \toprule
        Step & Variabile & & Valore \\ 
        \midrule
                0 & $\beta_{0}$ & $=$ &$8563$\\
        &$\delta$ & $=$ & ${8563^{28702/{127}}}$\\
        &$a_{0}$ & $=$ & $99$\\
        &$\beta_{1}$ & $=$ & $5766$\\
        \bottomrule
    \end{tabular}\end{center}

\textbf{N.B.} Il calcolo di $a$ dalle variabili $a_i$ ottenute dalle diverse iterazioni è stato omesso in quanto banale ($ a= \sum_{i = 0}^{c-1}a_iq^i$)
Ponendo le soluzioni ottenute a sistema si ottiene:

\begin{equation}
    \label{eq:1}
    \begin{cases}
        a \equiv 67 \mod 113\\
        a \equiv 1 \ \mod 2\\
        a \equiv 99 \mod 127
    \end{cases}
\end{equation}
Per il teorema cinese del resto la $a$ cercata si calcola con i seguenti passaggi. $n = 28702$, $\forall i = 1,\dots,k$ 
\[
    y_i = \frac{n}{n_i} \text{ ($n_i$ fattore di $n$)}
\]
\noindent
Per cui $y_0 = \frac{28702}{113} = 254,\ y_1 = \frac{28702}{2} = 14351,\ y_2 = \frac{28702}{127} = 226,$ successivamente si calcolano le inverse moltiplicative modulo $n_i$ ($z_i \equiv {y_{i}}^{-1}$). Si ottiene:
$z_0 = 109,\ z_1 = 1,\ z_2 = 68$.
Secondo il teorema 
\[
    a = \sum_{i=0}^{k}a_iy_iz_i
\]
$a$ è soluzione del sistema (\ref{eq:1})\\
$a = (67 \cdot 254 \cdot 109 ) + ( 1 \cdot 14351 \cdot 1) +  ( 99 \cdot 226 \cdot 68) = \fbox{3909 mod 28702}$
