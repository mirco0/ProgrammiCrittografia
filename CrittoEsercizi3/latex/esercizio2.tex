\textit{Birthday ``paradox'' con probabilità qualsiasi} Dimostrare se $\Omega$ è un insieme di $N$ elementi dotato
di una probabilità possibilmente non uniforme, allora dopo $3\lceil\sqrt{N}\rceil$ estrazioni indipendenti di
elementi da $\Omega$, la probabilità di aver estratto elementi tutti diversi è minore di 1/2.

\subsection{Soluzione}
Sia $\Omega = \{x_1,\dots,x_n\}$, con una distribuzione non uniforme $\forall i = 1,\dots, n,\ Pr[x_i] = p_i$ e $\sum_{i=1}^{n}{p_i} = 1$.  
Sia $Q = 3\lceil \sqrt{n}\rceil$ il numero di estrazioni.  

Si definisce $E_{i,j}$ l'evento in cui l'estrazione \textit{i-esima} e \textit{j-esima} $(X_i, X_j)$ generano una collisione:
\[
    Pr[E_{i,j}] = \sum_{k=1}^{n} Pr[X_i = k \land X_j = k] 
    = \sum_{k=1}^{n} Pr[X_i = k] \cdot Pr[X_j=k] 
    = \sum_{k=1}^{n} {p_k}^{2}.
\]

\noindent
Si ricorda lo \textit{union bound}:
\[  \boxed{
        Pr\Big[\bigcup_{i=1}^{n} A_i\Big] \leq \sum_{i=1}^{n} Pr[A_i]
    }
\]
\vspace{0.5cm}
\noindent
Sia ora $E$ l'evento di almeno una collisione tra le $Q$ estrazioni:
\[
    Pr[E] = Pr\Big[\bigcup_{1\leq i < j \leq Q} E_{i,j}\Big].
\]

Il numero di coppie $(i,j)$ con $1 \leq i < j \leq Q$ è $\binom{Q}{2}$, quindi applicando lo union bound:
\[
    Pr[E] \leq \binom{Q}{2} \sum_{k=1}^{n} p_k^{2}.
\]

\noindent
La quantità $\sum_{k=1}^{n} p_k^2$ è minimizzata quando la distribuzione è uniforme:
\[
    \sum_{k=1}^{n} p_k^2 \geq n \left(\frac{1}{n}\right)^2 = \frac{1}{n}.
\]

Quindi:
\[
    Pr[E] \leq \binom{Q}{2} \cdot \frac{1}{n}.
\]

L'evento complementare $E^c$ è l'evento che non avvengano collisioni:
\[
    Pr[E^c] = 1 - Pr[E]
    \leq 1 - \binom{Q}{2} \sum_{k=1}^{n} p_k^{2}.
\]

Usiamo la disuguaglianza $1-x \leq e^{-x}$:
\[
    Pr[E^c] \leq e^{-\binom{Q}{2} \sum_{k=1}^n p_k^2}.
\]

Sostituendo il minimo valore possibile di $\sum p_k^2$:
\[
    Pr[E^c] \leq e^{-\frac{Q(Q-1)}{2n}}.
\]

Ora poniamo $Q = 3\lceil \sqrt{n} \rceil$. Per $n$ sufficientemente grande, $Q(Q-1) \geq 9n$, dunque:
\[
    Pr[E^c] \leq e^{- \frac{9n}{2n}} = e^{-9/2}.
\]

Poiché $e^{-9/2} < \frac{1}{2}$, si conclude:
\[
    Pr[E^c] < \frac{1}{2}.
\]

\noindent
Dunque dopo $3\lceil \sqrt{N} \rceil$ estrazioni, la probabilità di non avere collisioni è minore di $1/2$. $\square$
