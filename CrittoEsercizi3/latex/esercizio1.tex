\subsection{7.1 Stinson-Paterson}
Implementa l'algoritmo di Shanks per trovare il logaritmo discreto in ${\mathbb{Z}_p}^*$. Con $p$ primo e $\alpha$ un elemento primitivo modulo $p$. Trova $\log_{106}(12375)$ in ${\mathbb{Z}_{24691}}^*$e $\log_6(248388)$ in ${\mathbb{Z}_{458009}}^*$

\subsection*{Soluzione}
L'implementazione dell'algoritmo di Shanks richiede anche l'implementazione di \textit{Campi} e \textit{Gruppi} (nello specifico moltiplicativi). Per semplicità il codice scritto tiene conto solamente dei gruppi moltiplicativi includendo le funzioni dei campi finiti. Segue una breve descrizione delle funzioni scritte
per la classe \texttt{Gruppo} modulo $p$ (numero primo)
\begin{itemize}
    \item \texttt{exp(a,b)} Per l'esponenziazione $a^b$ utilizzando la tecnica dello \textit{Square and Multiply}
    \item \texttt{mul(a,b)} Per l'operazione di moltiplicazione $a \cdot b$
    \item \texttt{inverse(a)} Per il calcolo dell'\textit{inversa moltiplicativa} modulo $p$, calcolato con l'\textit{algoritmo Euclideo} 
\end{itemize}
\noindent
Con queste funzioni è banale mettere in pratica l'algoritmo di \textit{Shanks}, è possibile visionare il codice nel file \texttt{Crittoesercizi3/Esercizio 1/Shanks.py}\\
\noindent
\textbf{N.B.} L'algoritmo seguente fa uso di dizionari (strutture dati, che in python hanno una complessità di \textit{lookup} $O(1)^*$)\footnote{Costo ammortizzato}

\subsubsection*{Calcolo logartimo discreto}
Qui segue l'esecuzione dell'algoritmo per il primo punto dell'esercizio:
Si ha $\alpha  = 106,n = 24690, \beta= 12375$, per cui $m = \lceil \sqrt{24691} \rceil = 158$. 
Proseguendo $\alpha^{158} \mod 24691 = 24689$.
\newpage
Si calcolano le coppie $(j,24689^j) \; \forall j = 0,\dots,m$, sono riportate nella tabella sottostante
\begin{center}
	\begin{tabular}[h]{*{7}{c}}
    (0, 1)&(1, 24689)&(2, 4)&(3, 24683)&(4, 16)&(5, 24659)&(6, 64)\\
    (7, 24563)&(8, 256)&(9, 24179)&(10, 1024)&(11, 22643)&(12, 4096)&(13, 16499)\\
    (14, 16384)&(15, 16614)&(16, 16154)&(17, 17074)&(18, 15234)&(19, 18914)&(20, 11554)\\
    (21, 1583)&(22, 21525)&(23, 6332)&(24, 12027)&(25, 637)&(26, 23417)&(27, 2548)\\
    (28, 19595)&(29, 10192)&(30, 4307)&(31, 16077)&(32, 17228)&(33, 14926)&(34, 19530)\\
    (35, 10322)&(36, 4047)&(37, 16597)&(38, 16188)&(39, 17006)&(40, 15370)&(41, 18642)\\
    (42, 12098)&(43, 495)&(44, 23701)&(45, 1980)&(46, 20731)&(47, 7920)&(48, 8851)\\
    (49, 6989)&(50, 10713)&(51, 3265)&(52, 18161)&(53, 13060)&(54, 23262)&(55, 2858)\\
    (56, 18975)&(57, 11432)&(58, 1827)&(59, 21037)&(60, 7308)&(61, 10075)&(62, 4541)\\
    (63, 15609)&(64, 18164)&(65, 13054)&(66, 23274)&(67, 2834)&(68, 19023)&(69, 11336)\\
    (70, 2019)&(71, 20653)&(72, 8076)&(73, 8539)&(74, 7613)&(75, 9465)&(76, 5761)\\
    (77, 13169)&(78, 23044)&(79, 3294)&(80, 18103)&(81, 13176)&(82, 23030)&(83, 3322)\\
    (84, 18047)&(85, 13288)&(86, 22806)&(87, 3770)&(88, 17151)&(89, 15080)&(90, 19222)\\
    (91, 10938)&(92, 2815)&(93, 19061)&(94, 11260)&(95, 2171)&(96, 20349)&(97, 8684)\\
    (98, 7323)&(99, 10045)&(100, 4601)&(101, 15489)&(102, 18404)&(103, 12574)&(104, 24234)\\
    (105, 914)&(106, 22863)&(107, 3656)&(108, 17379)&(109, 14624)&(110, 20134)&(111, 9114)\\
    (112, 6463)&(113, 11765)&(114, 1161)&(115, 22369)&(116, 4644)&(117, 15403)&(118, 18576)\\
    (119, 12230)&(120, 231)&(121, 24229)&(122, 924)&(123, 22843)&(124, 3696)&(125, 17299)\\
    (126, 14784)&(127, 19814)&(128, 9754)&(129, 5183)&(130, 14325)&(131, 20732)&(132, 7918)\\
    (133, 8855)&(134, 6981)&(135, 10729)&(136, 3233)&(137, 18225)&(138, 12932)&(139, 23518)\\
    (140, 2346)&$\mathbf{(141, 19999)}$&(142, 9384)&(143, 5923)&(144, 12845)&(145, 23692)&(146, 1998)\\
    (147, 20695)&(148, 7992)&(149, 8707)&(150, 7277)&(151, 10137)&(152, 4417)&(153, 15857)\\
    (154, 17668)&(155, 14046)&(156, 21290)&(157, 6802)&
	\end{tabular}
\end{center}
La seconda lista contiene le coppie $(i,12375 \cdot 106^{-i} \mod 24691) \; \forall i = 0,\dots,m$ 
\begin{center}
	\begin{tabular}[h]{*{7}{c}}
    (0, 12375)&(1, 20382)&(2, 18827)&(3, 19977)&(4, 1819)&(5, 14692)&(6, 23432)\\
    (7, 3948)&(8, 7957)&(9, 308)&(10, 10252)&(11, 6153)&(12, 7279)&(13, 23595)\\
    (14, 2319)&(15, 20753)&(16, 6485)&(17, 7748)&(18, 14515)&(19, 24595)&(20, 14441)\\
    (21, 11550)&(22, 14085)&(23, 20864)&(24, 13707)&(25, 8282)&(26, 544)&(27, 471)\\
    (28, 8623)&(29, 18949)&(30, 13456)&(31, 21091)&(32, 10681)&(33, 13378)&(34, 22022)\\
    (35, 16979)&(36, 7847)&(37, 14283)&(38, 20400)&(39, 5317)&(40, 14725)&(41, 6894)\\
    (42, 10780)&(43, 13146)&(44, 17827)&(45, 7855)&(46, 11022)&(47, 7092)&(48, 10316)\\
    (49, 4756)&(50, 24270)&(51, 14205)&(52, 21331)&(53, 11615)&(54, 9194)&(55, 23846)\\
    (56, 14201)&(57, 10616)&(58, 18269)&(59, 19040)&(60, 16485)&(61, 5513)&(62, 21249)\\
    (63, 1831)&(64, 22146)&(65, 3470)&(66, 23792)&(67, 5349)&(68, 1681)&(69, 3044)\\
    (70, 6085)&(71, 22652)&(72, 13258)&(73, 21555)&(74, 19071)&(75, 6935)&(76, 15672)\\
    (77, 6670)&(78, 3324)&(79, 15405)&(80, 1310)&(81, 15386)&(82, 18314)&(83, 9956)\\
    (84, 3355)&(85, 5855)&(86, 11469)&(87, 807)&(88, 19807)&(89, 3215)&(90, 1195)\\
    (91, 22140)&(92, 24434)&(93, 9082)&(94, 20118)&(95, 2985)&(96, 14703)&(97, 9689)\\
    (98, 22686)&(99, 17917)&(100, 14378)&(101, 9453)&(102, 7776)&(103, 15447)&(104, 2708)\\
    (105, 19592)&(106, 13695)&(107, 828)&(108, 20506)&(109, 5318)&(110, 11231)&(111, 17576)\\
    (112, 20664)&(113, 21159)&$\mathbf{(114, 19999)}$&(115, 23715)&(116, 2786)&(117, 18661)&(118, 7397)\\
    (119, 6359)&(120, 3554)&(121, 1897)&(122, 13761)&(123, 17134)&(124, 9479)&(125, 15696)\\
    (126, 21578)&(127, 12782)&(128, 5711)&(129, 20785)&(130, 18132)&(131, 3898)&(132, 9820)\\
    (133, 9410)&(134, 9872)&(135, 559)&(136, 22134)&(137, 20707)&(138, 19063)&(139, 10196)\\
    (140, 4289)&(141, 1671)&(142, 13293)&(143, 22720)&(144, 22576)&(145, 7201)&(146, 24526)\\
    (147, 8617)&(148, 15222)&(149, 23437)&(150, 11169)&(151, 11985)&(152, 346)&(153, 935)\\
    (154, 17013)&(155, 12506)&(156, 7106)&(157, 10782)&
	\end{tabular}
\end{center}
Da queste due tabelle si notano le coppie $(141,19999)$ e $(114,19999)$ che hanno una collisione. Per cui il valore del logaritmo discreto cercato è
\[
    \log_{106}(12375) = (158 \cdot 141 + 114) \mod 24691 = \fbox{22392}
\]
\newpage
Analogamente si procede con il secondo punto dell'esercizio, ovvero il calcolo di $\log_6(248388)$ in ${\mathbb{Z}_{458009}}$, non presentando in modo esplicito le coppie, ma brevemente i passaggi.\\
Si ha $\alpha = 6,n = 458009, \beta= 248388$, per cui $m = \lceil \sqrt{458009} \rceil = 677$. Per cui $\alpha^{677} \mod 458009 = 186889$. In questo caso le coppie che conicidono sono $(343, 157943)$ e $(625, 157943)$, il valore del logaritmo discreto cercato è
\[
    \log_{6}(248388) = (677 \cdot 343 + 625) \mod 458009 = \fbox{232836}
\]

\subsection{7.5 Stinson-Paterson}
Implementa l'algoritmo Pohlig-Hellman per trovare il logaritmo discreto in ${\mathbb{Z}_p}^*$. Con $p$ primo e $\alpha$ un elemento primitivo modulo $p$. Trova $\log_{5}(8563)$ in ${\mathbb{Z}_{28703}}$ e $\log_{10}(12611)$ in ${\mathbb{Z}_{31153}}$

\subsection*{Soluzione}
Per procedere con l'implementazione dell'algoritmo sono necessarie alcune funzioni correttamente implementate:
\begin{itemize}
    \item Algoritmo di fattorizzazione
    \item Algoritmo di \textit{Shanks} per il calcolo di logaritmi discreti
\end{itemize}

L'algoritmo di fattorizzazione è visibile nel file \texttt{Crittoesercizi3/Esercizio 1/PohligHellman.py}, per semplicità e stato scritto in modo ricorsivo, e fa uso della programmazione dinamica per il calcolo immediato della fattorizzazione per input precedentemente visti.\\
L'output è una lista di tuple $(p_0,c_0),\dots,(p_k,c_k)$ tale che  
\[
    n = \prod_{i = 0}^{k}{ {p_i}^{c_i}}
\]

Una volta ottenuta la fattorizzazione di $n$, il resto dell'implementazione, segue passo passo la teoria. Difatti il codice principale fa uso della fattorizzazione di $n$ per il calcolo di logaritmi discreti $a = \log_{\alpha}(\beta)$\\
con $a\mod {p_i}^{c_i} \; \forall i = 1,\dots,k$.

\subsubsection*{Calcolo logartimo discreto}
Si descrivono brevemente i dati del primo punto dell'esercizio $\log_{5}(8563)$ in ${\mathbb{Z}_{28703}}$, per cui $\alpha = 5,n = 28702, \beta = 8563$. La fattorizzazione di $n$ è $28702 = 113^1 \cdot 127^1 \cdot 2^1$.

\noindent
Test per il fattore $q = 113$ $c = 1$
\begin{center}
    \begin{tabular}{@{}lccl@{}}
        \toprule
        Step & Variabile & & Valore \\ 
        \midrule
                0 & $\beta_{0}$ & $=$ &$8563$\\
        &$\delta$ & $=$ & ${8563^{28702/{113}}}$\\
        &$a_{0}$ & $=$ & $67$\\
        &$\beta_{1}$ & $=$ & $17744$\\
        \bottomrule
    \end{tabular}\end{center}
Test per il fattore $q = 2$ $c = 1$
\begin{center}
    \begin{tabular}{@{}lccl@{}}
        \toprule
        Step & Variabile & & Valore \\ 
        \midrule
                0 & $\beta_{0}$ & $=$ &$8563$\\
        &$\delta$ & $=$ & ${8563^{28702/{2}}}$\\
        &$a_{0}$ & $=$ & $1$\\
        &$\beta_{1}$ & $=$ & $24675$\\
        \bottomrule
    \end{tabular}\end{center}
Test per il fattore $q = 127$ $c = 1$
\begin{center}
    \begin{tabular}{@{}lccl@{}}
        \toprule
        Step & Variabile & & Valore \\ 
        \midrule
                0 & $\beta_{0}$ & $=$ &$8563$\\
        &$\delta$ & $=$ & ${8563^{28702/{127}}}$\\
        &$a_{0}$ & $=$ & $99$\\
        &$\beta_{1}$ & $=$ & $5766$\\
        \bottomrule
    \end{tabular}\end{center}

\textbf{N.B.} Il calcolo di $a$ dalle variabili $a_i$ ottenute dalle diverse iterazioni è stato omesso in quanto banale ($ a= \sum_{i = 0}^{c-1}a_iq^i$)
Ponendo le soluzioni ottenute a sistema si ottiene:

\begin{equation*}
    \begin{cases}
        a \equiv 67 \mod 113\\
        a \equiv 1 \ \mod 2\\
        a \equiv 99 \mod 127
    \end{cases}
\end{equation*}
Per il teorema cinese del resto la $a$ cercata si calcola con i seguenti passaggi. $n = 31152$, $\forall i = 1,\dots,k$ 
\[
    y_i = \frac{n}{n_i} \text{ ($n_i$ fattore di $n$)}
\]
\noindent
Per cui $y_0 = \frac{28702}{113} = 254,\ y_1 = \frac{28702}{127} = 226,\ y_2 = \frac{28702}{2} = 14351$ successivamente si calcolano le inverse moltiplicative modulo $n_i$ ($z_i \equiv {y_{i}}^{-1}$). Si ottiene:
$z_0 = 109,\ z_1 = 68,\ z_2 = 1$.

