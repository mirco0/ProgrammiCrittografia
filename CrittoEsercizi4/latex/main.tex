\documentclass{article}
\usepackage[margin=2cm]{geometry}
\usepackage[utf8]{inputenc}
\usepackage[italian]{babel}
\usepackage[shortlabels]{enumitem}
\usepackage{amssymb}
\usepackage{amsmath}
\usepackage{mdframed}

\title{Esercizi Crittografia}
\date{12 gennaio 2025}
\author{Mirco Pasquali}
\begin{document}

\maketitle
\input{matricola.tex}
\begin{quote}
    Nelle soluzioni degli esercizi faccio riferimento a definizioni e teoremi dai libri consigliati dove possibile, in altri casi potrei fare riferimento a definizioni trovate su \emph{Wikipedia}
\end{quote}

\section{Esercizio 1}
\subsection{Esercizio 3}
Sia $E: y^2 = x^3 + cx^2 + dx + e$ una curva ellittica su un campo $k$. Calcolare il grado delle mappe $E \to \mathbb{P}^1$ date dalle seguenti funzioni razionali:
\begin{enumerate}[(a)]
    \item $f(x,y) = x$
    \item $f(x,y) = y$
    \item $f(x,y) = y-x$
\end{enumerate}

\subsection*{Soluzione}
Ricordo la definizione di grado di una mappa tra campi dal libro \emph{Silverman}
    \begin{mdframed}
        Sia $\phi: C_1 \to C_2$ una mappa di curve definite su $K$. Se $\phi$ è non costante, è detta \emph{mappa finita}, di cui il grado è
        \[
            \deg \phi = \left[ K(C_1) : \phi^*K(C_2)\right]
        \]
    \end{mdframed}
    Per cui il grado della mappa è il grado dell'estensione di campo $K(C_1)$ su $\phi^*K(C_2)$, dalla teoria, un polinomio irriducibile $p \in K(x)$ in un campo $K$, genera un campo d'estensione contente l'elemento algebrico $\alpha$ radice di $p$. Il grado del polinomio minimo che genera tale estensione è proprio il grado dell'estensione.

    \vspace{1em}\noindent
    Per $f(x,y) = x$ si deve calcolare $[K(E):K(x)]$ il grado di estensione del campo di funzioini sulla curva sul campo della retta proiettiva, l'equazione della curva è
    \[
        y^2 = x^3 + cx^2 + dx + e
    \]
    Consideriamo questo polinomio come un'equazione a variabile $y$, nel campo $K(x)$ ($K(x)[y]$). Se il polinomio $L(x) = x^3 + cx^2 + dx +e$ fosse un quadrato su $K(x)$ (composto da funzioni razionali).
    \[
        L(x) = \frac{p(x)}{q(x)}
    \]
    Il grado di $L$ è dato da $\deg(p) - \deg(q)$, se $L(x)$ fosse un quadrato allora
    \[
        L(x) = \frac{p(x)^2}{q(x)^2}
    \]
    $L(x)$ sarebbe un polinomio con grado $2\deg(p) - 2\deg(q)$, ovvero con grado pari. Per cui $L(x)$ non può essere un quadrato, e $y^2 - L(x)$ è un polinomio irriducibile, $y$ è algebrico su $K(x)$, il grado di estensione è uguale al grado rispetto a $y$ del polinomio minimo $y^2 - L(x)$, per cui il grado di estensione è 2.

    
    \vspace{2em}\noindent
    Per $f(x,y) = y$, si deve calcolare il grado $[K(E):K(y)]$. Si ha il polinomio $P(X)$ in $K(y)[X]$, 
    \[
        P(X) = X^3 +cX^2 +dX + e - y^2
    \]

    $P(X)$ non è riducibile, non può esistere $X_0 \in K(y)$ tale che $P(X_0) = 0$.$x = X_0$ sarebbe una funzione razionale di $y$, e
    \[
        y^2 = x^3 + c x^2 + d x + e
    \] 
    sarebbe un quadrato di una funzione in $y$ uguale a un polinomio di grado 3, quindi $P(X)$ è irriducibile, $X$ è algebrico su $K(y)$ e come prima, il grado di estensione è uguale al grado del polinomio minimo $P(X)$, il grado di estensione è 3.

    \vspace{2em}\noindent
    Per $f(x,y) = y-x$ si deve calcolare il grado $[K(E):K(y-x)]$. Data l'equazione della curva, e fissando una $t = y - x$ quindi sostituendo $y = t + x$,  si ottiene un polinomio 
    \[
        (t + x)^2 = x^3 + cx^2 + dx + e \qquad  x^3 + (c-1)x^2 + (d-2t)x + e - t^2 = 0
    \]
    ovvero un polinomio di terzo grado in $K(t)$, irriducibile come nel caso precedente, per cui il grado di estensione è 3.

% \newpage
\section{Esercizio 2}
Sia $E$ una curva ellittica definita su un campo finito $\mathbb{F}_q$. Dimostrare i seguenti fatti sul gruppo dei punti $E(\mathbb{F}_q)$.
\begin{enumerate}[(a)]
    \item Se la curva $E$ è data da un'equazione della forma $y^2 = x(x - 1)(x - \lambda)$ per qualche $\lambda \in \mathbb{F}_q$,
        allora $E(\mathbb{F}_q)$ non è ciclico.

    \item In generale, esistono interi positivi $n, k$ tali che $E(\mathbb{F}_q)$ è isomorfo a $(\mathbb{Z}/n\mathbb{Z}) \times (\mathbb{Z}/k\mathbb{Z})$ (è
    compresa la possibilità $k = 1$, in cui si ha $(\mathbb{Z}/n\mathbb{Z}) \times (\mathbb{Z}/k\mathbb{Z}) \cong (\mathbb{Z}/n\mathbb{Z}))$
\end{enumerate}
\subsection*{Soluzione 5a}
%Pag 77 libro.
Dalla teoria, su una curva ellittica un punto $P$ soddisfa $2P = O_E$, se e solo se la tangente in $P$ è verticale. Sia la curva ellittica definita da un'equazione cubica, $y^2 = (x-e_1)(x-e_2)(x-e_3)$, i punti  $P$ che soddisfano $2P = O_E$ sono proprio quelli per $y = 0$.

\vspace{1em}\noindent
Per l'equazione data $y^2 = x(x-1)(x-\lambda)$, si ottiene il gruppo di punti di due torsione $E[2] = \{ O_E,(0,0),(1,0),(\lambda,0)\}$ di ordine 4, in cui gli elementi $(0,0),(1,0),(\lambda,0)$ sono di ordine 2, verificabile anche attraverso un semplice calcolo

\vspace{1em}
L'equazione della curva data è equivalente a
\[
    y^2 = x^3 + (\lambda-1) x^2 + \lambda x 
\]
Ovvero una curva ellittica in forma generalizzata di Weierstass $y^2 + a_1xy + a_3y = x^3 + a_2x^2 + a_4x + a_6$, con i coefficienti
\[
    a_1 = 0 \quad a_2 = \lambda - 1 \quad a_3 = 0 \quad a_4 = \lambda \quad a_6 = 0
\]

Data l'equazione di Weierstass generalizzata è possibile calcolare il punto $-P$ relativo a un punto $P = (x,y)$ con la formula $-P(x,-a_1x-a_3-y)$, per cui nel nostro caso si ha che $-P = (x,-y)$.\\
Si ottiene $\forall P \in \{ (0,0),(1,0),(\lambda,0)\} \ P = -P$ e $2P = P + P = P + (-P) = O_E$, i punti sono di due torsione e generano i sottogruppi di ordine 2
\begin{align*}
    \langle (0,0) \rangle = \{ O_E, (0,0)\} \quad
    \langle (1,0) \rangle = \{ O_E , (1,0)\} \quad
    \langle (\lambda,0) \rangle = \{ O_E, (\lambda, 0)\} 
\end{align*}
%Pag 478 https://en.wikipedia.org/wiki/Subgroups_of_cyclic_groups
Per definizione un gruppo $G$ di ordine $n$ è ciclico se e solo se per ogni divisore $d$ di $n$, $G$ ha al massimo un sottogruppo di ordine $d$.
\vspace{1em}
In questo caso il gruppo $E(\mathbb{F}_q)$ non è ciclico perché ha $3$ elementi di ordine $2$, che generano $3$ sottogruppi di ordine $2$.

\subsection*{Soluzione 5b}

Dalla teoria, per la \emph{legge di gruppo} sulle curve ellittiche, $E(\mathbb{F}_q)$ è un gruppo abeliano finito.

Per il teorema fondamentale dei gruppi abeliani finiti, $E(\mathbb{F}_q)$ è isomorfo a una somma diretta di gruppi ciclici
\[
    E(\mathbb{F}_q) \cong \mathbb{Z}_{n_1} \oplus \mathbb{Z}_{n_2} \oplus \cdots \oplus \mathbb{Z}_{n_r}
\]
in cui $n_1 \mid n_2 \mid \cdots \mid n_r$.

In cui in ogni gruppo $n_i$ ha $n_1$ elementi di ordine che divide $n_1$.

\vspace{1em}\noindent
Quindi $E(\mathbb{F}_q)$ è isomorfo alla somma diretta di $r$ gruppi ciclici. Da un teorema noto, per una curva ellittica su un campo $K$ e un intero $n$, se la caratteristica di $K$ non divide $n$. 
\[
    E[n] \cong \mathbb{Z}_n \oplus \mathbb{Z}_n.
\]
\noindent
Il sottogruppo degli elementi di ordine che divide $n_1$ ha cardinalità $n_1^r$ ed è contenuto in $E[n_1]$, ma $E[n_1]$
%$\mathbb{Z}_{n_1} \oplus \mathbb{Z}_{n_2} \oplus \cdots \oplus \mathbb{Z}_{n_r}$
contiene al massimo $n_1^2$ elementi. 

\vspace{1em}
Poiché $E(\mathbb{F}_q) \cong \mathbb{Z}_{n_1} \oplus \mathbb{Z}_{n_2} \oplus \cdots \oplus \mathbb{Z}_{n_r}$ allora
\[
    {n_1}^r \leq {n_1}^2 \quad r \leq 2
\]
quindi $E(\mathbb{F}_q)$ ha al più due fattori ciclici per due interi $n,k$. 

Il gruppo dei punti razionali di una curva ellittica su un campo finito ha al più due fattori ciclici, quindi esistono interi positivi $n,k$ tali che
\[
    E(\mathbb{F}_q) \cong \mathbb{Z}_n \times \mathbb{Z}_k.
\]

\end{document}