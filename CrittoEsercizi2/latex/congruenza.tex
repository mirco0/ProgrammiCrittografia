\[
P(x)=x^8+x^4+x^3+x+1, \qquad 
x^8 \equiv x^4+x^3+x+1 \pmod{P}.
\]

Il nostro obiettivo è mostrare che:
\[
x^{256} \equiv x \pmod{P}.
\]

\section*{Calcolo delle potenze successive}

\subsection*{1. Potenza \(x^{16}\)}
\[
x^{16} = (x^8)^2 \equiv (x^4+x^3+x+1)^2.
\]
Sviluppando:
\[
(x^4+x^3+x+1)^2 = x^8 + x^6 + x^2 + x.
\]
Sostituendo \(x^8 \equiv x^4+x^3+x+1\):
\[
x^{16} \equiv x^6 + x^4 + x^3 + x^2 + x.
\]

\subsection*{2. Potenza \(x^{32}\)}
\[
x^{32} = (x^{16})^2 = (x^6+x^4+x^3+x^2+x)^2.
\]
Dopo l'espansione e la riduzione modulo \(P\):
\[
x^{32} \equiv x^7 + x^6 + x^5 + x^2.
\]

\subsection*{3. Potenza \(x^{64}\)}
\[
x^{64} = (x^{32})^2 = (x^7+x^6+x^5+x^2)^2.
\]
Riducendo:
\[
x^{64} \equiv x^6 + x^3 + x^2 + 1.
\]

\subsection*{4. Potenza \(x^{128}\)}
\[
x^{128} = (x^{64})^2 = (x^6+x^3+x^2+1)^2.
\]
Semplificando:
\[
x^{128} \equiv x^7 + x^6 + x^5 + x^4 + x^3 + x.
\]

\subsection*{5. Potenza finale \(x^{256}\)}
\[
x^{256} = (x^{128})^2 = (x^7+x^6+x^5+x^4+x^3+x)^2.
\]
Effettuando l'espansione e applicando la sostituzione per \(x^8\):
\[
\boxed{
x^{256} \equiv x \pmod{P}.
}
\]

\section*{Nota sulle riduzioni intermedie}
Esempi utili:
\[
x^{12} = x^4 x^8 \equiv x^4 (x^4+x^3+x+1)
        = x^7 + x^5 + x^4 + x^3 + x,
\]
\[
x^{14} = x^{12}x^2, \qquad
x^{10} = x^8 x^2.
\]

